\chapter{INTRODUÇÃO}
A arquitetura de software é um processo complexo e dinâmico. Conforme Martin (2020)
afirma, "uma arquitetura bem definida é um processo contínuo de investigação, e não um artefato
congelado". Ou seja, trata-se de um estudo contínuo e análise profunda do software, bem como
das definições e previsões futuras. Esse processo é condicionado a uma operação constante, em
que todos os envolvidos no software devem estar alinhados com as tecnologias e as definições
do profissional responsável pela arquitetura e definições técnicas do software. Esse processo
de desenhar, planejar e construir uma aplicação é contínuo e envolve decisões críticas para
o software. Muitas vezes, é referido como design de software, pois envolve a definição de
componentes e conectores entre as aplicações do sistema. Além disso, é necessário estabelecer
padrões de sistema, como a arquitetura em camadas, orientada a eventos, entre outras. Para
arquitetar um software, é fundamental ter uma visão de longo prazo, prevendo futuras abordagens
e qualidades do software.

\section{Contextualização}
A definição das arquiteturas para aplicações torna-se cada vez mais complexa, pois muitas
decisões iniciais podem não estar claras, como a quantidade de acessos e o público-alvo. No
entanto, é possível definir aspectos importantes desde o início, como se a comunicação será
síncrona ou assíncrona, e onde os dados serão armazenados. Para isso, é necessário arquitetar,
modelar e redesenhar o sistema várias vezes até que atenda às expectativas do arquiteto ou às
necessidades do produto.

\section{Delimitação do problema}

A implementação de uma arquitetura robusta de software pode ser um desafio financeiro,
onde o papel do arquiteto de software está mais voltado para a concepção da aplicação e não gera
lucro direto. Considerando isso, o objetivo deste trabalho é destacar a importância crítica e a ne-
cessidade de uma arquitetura de software apropriada para o desenvolvimento de um protótipo de
aplicação de envio de documentos, com foco em uma arquitetura de micro serviços, mensageria
e modelando com C4 Model. Além disso, essa pesquisa pretende examinar e incorporar modelos
de arquitetura sugeridos por profissionais da comunidade e por cientistas proeminentes da área,
visando simplificar a procura por soluções e minimizar os gastos relacionados à arquitetura de
software. O estudo também tem como meta entender as razões que levam startups a hesitar em
investir em arquitetos de software, fornecendo um modelo econômico e eficaz para a aplicação
de envio de documentos.

\section{Objetivos}

\subsection{Objetivo geral}
Desenvolver o protótipo de aplicação utilizando arquitetura de micro serviços, mensageria
e modelagem com C4 model.

\subsection{Objetivos específicos}
\begin{itemize}
    \item Desenvolver orientado à arquitetura de software utilizando a modelagem C4 Model;
    \item Explorar técnicas e tecnologias modernas para o desenvolvimento;
\end{itemize}

\section{Justificativa}
A medida que o volume de documentos digitais aumenta exponencialmente nas empresas
e instituições, torna-se cada vez mais crítico possuir um sistema eficiente de envio de documentos.
Além disso, a crescente demanda por escalabilidade, disponibilidade e confiabilidade nos sistemas
de TI torna a arquitetura de micro serviços, mensageria e modelagem com C4 Model uma escolha
apropriada para o desenvolvimento desta aplicação.

O uso da arquitetura de micro serviços permitirá modularizar a aplicação, separando as
funções e responsabilidades em serviços independentes. Isso facilita a manutenção e escalabi-
lidade do sistema, além de garantir maior resiliência, já que a falha de um serviço não afeta a
operação dos outros.

A inclusão da mensageria na arquitetura auxilia na comunicação assíncrona entre os
micro serviços, permitindo um maior desacoplamento entre eles. Esse desacoplamento garante
uma melhor distribuição de carga e uma resposta mais rápida às mudanças, tanto em termos de
volume de tráfego quanto de alterações nos requisitos do sistema.

O uso do C4 Model, que é uma estrutura para visualização da arquitetura de software,
oferece uma maneira clara e intuitiva de descrever e comunicar a estrutura do sistema. Essa
abordagem melhora a colaboração entre os membros da equipe, permitindo uma compreensão
comum da arquitetura e facilitando a tomada de decisões de design.

A proposta deste trabalho é desenvolver um protótipo de uma aplicação que seja escalável,
resiliente e de fácil manutenção, que se beneficie das vantagens das tecnologias de micro
serviços, mensageria e C4 Model, e que possa eficientemente gerenciar e realizar o envio de
documentos. Considerando as necessidades crescentes das empresas para gerir grandes volumes
de documentos digitais, acredita-se que este trabalho terá um impacto significativo na eficiência
das operações de TI e na produtividade geral das organizações.

\section{Procedimentos metodológicos}
O trabalho inicia com a definição clara do problema e das necessidades que a aplicação
de envio de documentos deverá resolver. Isso envolve o alinhamento das expectativas, limitações,
desafios e objetivos do projeto.
Com o problema definido, prosseguimos para a modelagem dos dados usando o C4 Model.
Nesta etapa, são criados diagramas para visualizar a arquitetura da aplicação em diferentes níveis
de abstração, desde o contexto do sistema até os componentes individuais.
Baseando-se na modelagem de dados, identificamos as estruturas de micro serviços
necessárias e as comunicações entre eles. Este procedimento também engloba a definição da
formatação e dos estilos de comunicação que serão empregados.
Com os micro serviços e as comunicações definidos, começamos o desenvolvimento do
protótipo da aplicação. Garantimos que os micro serviços sejam bem definidos e capazes de se
comunicarem eficientemente.
Em seguida, implementamos a mensageria, um método de comunicação entre os micro
serviços. Escolhemos uma plataforma de mensageria adequada, com o Kafka, a configuramos e
a integramos aos micro serviços.
Depois da implementação, testamos a aplicação para garantir que o fluxo de envio de
documentos funciona conforme o esperado. Os testes podem incluir testes unitários, de integração
e de carga. Baseando-se nos resultados, realizamos os ajustes necessários.
Por fim, documentamos todo o processo, desde a definição do problema até os testes e
ajustes. A documentação descreve como a aplicação foi modelada, quais micro serviços foram
utilizados, como a mensageria foi implementada, e os resultados dos testes.


\section{Estrutura do trabalho}
Neste capítulo foram apresentados tópicos sobre arquitetura de software e introdução
acerca do trabalho e do conteúdo, nos próximos capítulos, citamos a revisão bibliográfica e eixos
norteadores do trabalho.