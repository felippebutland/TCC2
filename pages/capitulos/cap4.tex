\chapter{Considerações finais}
Monitoramento de esforço cognitivo pode ser um aliado na prevenção ou detecção precoce de cenários de estresse crônico e depressão. No entanto, o processo de identificação de esforço cognitivo, até então depende de métodos subjetivos e com reprodução inviável, o que impossibilita o seu monitoramento em escala, de forma acessível e não intrusiva.

Neste cenário a propõe-se o desenvolvimento de um modelo de inteligência artificial, treinado com dados de dispositivos vestíveis populares, capaz de identificar o esforço cognitivo. Este modelo será treinado com dados obtidos durante a realização de uma tarefa com alta demanda cognitiva e comparado com valores coletados por outros dispositivos dedicados para monitoramento de sinais psicofisiológicos, a fim de garantir a eficiência do modelo sem necessidade de um dispositivo de uso clínico.

Este desenvolvimento promove um avanço no processo de monitoramento de esforço cognitivo, uma vez que fornecerá um meio de detecção sem adicionar um equipamento extra nem requisitar uma intervenção manual para catalogar uma situação com maior demanda mental. 

O desenvolvimento do trabalho iniciou-se em fevereiro de 2024 e as demais etapas previstas estão detalhadas no cronograma quinzenal na Tabela \ref{ref:cronograma}. As etapas de pré-processamento, treinamento e validação acontecerão nos períodos de Preparação e tratamento de dados, Treinamento dos modelos e Avaliação dos modelos, respectivamente.  
\begin{table}
\centering
\tiny
\begin{threeparttable}
\caption{Cronograma quinzenal}
\label{ref:cronograma} 
\setlength\tabcolsep{5pt}
\begin{tabular}{ |l|c|c|c|c|c|c|c|c|c|c|c|c|c|c|c|c|c|c|c|c|c|c| } 
\hline
Atividade & \multicolumn{2}{|c|}{Fev} & \multicolumn{2}{|c|}{Mar} & \multicolumn{2}{|l|}{Abr} & \multicolumn{2}{|c|}{Mai} & \multicolumn{2}{|c|}{Jun} & \multicolumn{2}{|c|}{Jul} & \multicolumn{2}{|c|}{Ago} & \multicolumn{2}{|c|}{Set} & \multicolumn{2}{|c|}{Out} & \multicolumn{2}{|c|}{Nov} & \multicolumn{2}{|c|}{Dez} \\
\hline
Seleção de tema &X\tnote{1}&R\tnote{2}&P\tnote{3}&&&&&&&&&&&&&&&&&&& \\
\hline
Contextualização &X&&R&R&R&&&&&&&&&&&&&&&&&\\
\hline
Revisão bibliográfica &X&&R&R&R&R&R&P&P&P&P&P&P&P&P&P&P&P&P&P&&\\
\hline
Elaboração do projeto &X&&R&R&R&R&R&P&P&&&&&&&&&&&&&\\
\hline
Apresentação do projeto &X&&&&&&&&P&P&&&&&&&&&&&&\\
\hline
Preparação e tratamento de dados &X&&&&&&&&&&P&P&P&P&&&&&&&&\\
\hline
Treinamento dos modelos &X&&&&&&&&&&&&P&P&P&P&&&&&&\\
\hline
Avaliação dos modelos &X&&&&&&&&&&&&&&&&P&P&P&&&\\
\hline
Teste de implementação &X&&&&&&&&&&&&&&&&&&P&P&&\\
\hline
Entrega a banca &X&&&&&&&&&&&&&&&&&&&&P&\\
\hline
Documentação de atividades &X&&&&&&&&&&P&P&P&P&P&P&P&P&P&P&P&\\
\hline
Defesa da monografia &X&&&&&&&&&&&&&&&&&&&&P&P\\
\hline
\end{tabular}
\begin{tablenotes}
\item[1] Não incluso.
\item[2] Realizado.
\item[3] Previsto.
\end{tablenotes}
\end{threeparttable}
\sourcemedaddy
\end{table}

Por fim, a identificação de esforço cognitivo também pode ser usada para outros objetivos, como filtro de notificações, evitando distrações em algum momento com alta demanda mental, ou sinalizar a falta de esforço cognitivo ao realizar algum procedimento sensível, indicando uma possível distração que pode causar um acidente. Ademais, o método de treinamento pode auxiliar o desenvolvimento de outras tecnologias que buscam identificar estados mentais através do uso de sensores em dispositivos vestíveis.
