\chapter{Conclusão}

Desenvolver uma aplicação de envio de documentos, embora comum e aparentemente trivial, demanda uma arquitetura bem estruturada para evitar problemas em fases mais avançadas, conforme discutido nos capítulos anteriores. Neste trabalho, foi construído um protótipo de aplicação baseado em uma arquitetura de microserviços e orientado à mensageria, modelado utilizando o C4 Model, e que visou tanto a escalabilidade quanto a flexibilidade no atendimento das regras de negócio.

Uma arquitetura de software robusta é especialmente importante em aplicações empresariais que buscam escalar e garantir a eficiência na entrega de funcionalidades críticas. Contudo, como destacado na delimitação do problema, implementar uma arquitetura robusta pode ser financeiramente desafiador, especialmente para startups que frequentemente hesitam em investir em arquitetos de software. A importância de uma arquitetura bem planejada é frequentemente subestimada, pois seus benefícios, embora não gerem lucro direto, impactam diretamente a sustentabilidade, desempenho e manutenibilidade do sistema ao longo do tempo.

Este protótipo pode ser considerado bem-sucedido por atender aos principais requisitos de uma arquitetura baseada em microserviços, demonstrando escalabilidade e permitindo boa usabilidade no dia a dia. O teste de benchmark mostrou que a aplicação consegue lidar com uma quantidade satisfatória de requisições, validando a estrutura modular e o uso de mensageria como base para comunicação entre os serviços. No entanto, observamos que a fragmentação excessiva em microserviços, como a separação de login, cadastro e envio de documentos, pode introduzir latências adicionais, especialmente em ambientes multi-região, onde a localização geográfica dos serviços afeta o tempo de resposta.

A modelagem com o C4 Model também mostrou-se uma escolha acertada para documentar a arquitetura e suas interações, facilitando a visualização e o entendimento do sistema tanto para desenvolvedores quanto para stakeholders não técnicos. Esse tipo de documentação é vital para a continuidade do projeto, pois possibilita a disseminação de conhecimento entre os membros da equipe e evita a dependência de conhecimento tácito de indivíduos específicos. Em muitas equipes, a falta de documentação apropriada e a ausência de boas práticas de compartilhamento de conhecimento podem ser um grande obstáculo, especialmente na entrada de novos integrantes.

Além disso, ao explorar técnicas e tecnologias modernas, como microserviços e mensageria, este estudo mostrou a viabilidade de desenvolver uma solução econômica e eficaz, mesmo que seja apenas um protótipo. Foram seguidas práticas sugeridas por profissionais da área e cientistas, criando uma base sólida que simplifica a busca por soluções e minimiza os gastos associados a uma arquitetura de software sofisticada.

Em resumo, este trabalho enfatizou a importância de uma arquitetura de software bem planejada e apropriada para a criação de uma aplicação de envio de documentos. Ao implementar uma arquitetura de microserviços, utilizamos mensageria e modelagem com o C4 Model para construir um protótipo funcional que alcança um equilíbrio entre eficiência e simplicidade. Por fim, este estudo não só demonstrou a viabilidade econômica de um sistema robusto para startups, mas também propôs um modelo que pode auxiliar pequenas empresas a entenderem a importância estratégica de investir em arquiteturas bem definidas e na documentação de software.