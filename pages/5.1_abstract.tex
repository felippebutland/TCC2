% resumo em inglês
\begin{resumo}[Abstract]
 \begin{otherlanguage*}{english}
\noindent IGNAULIN, R.  \textbf{Prototype of an agricultural data recommendation system based on a scalable data platform}. 2023. Undergraduate Thesis – Computer Science, Universidade Comunitária da Região de Chapecó, Chapecó, 2023.\\

\noindent This undergraduate thesis aims to create a robust data platform to carry out exploratory analyzes involving the technical area of agronomy. For this, important concepts about application architectures were used, in addition to the application of cloud computing concepts for big data processing. Various types of data were centralized on a robust and scalable cloud storage platform. The platform chosen was AWS, which allowed the use of several services that are reliable and highly available in the cloud. Some open-source frameworks such as Apache Spark and Apache Airflow were used for distributed processing and orchestration of data pipelines. Various information related to meteorological data (INMET) was consumed, in addition to the collection of specific agricultural information, made available by (MAPA). The use of updated processes allows the processing and analysis of data in greater detail, important for generating insights. Several analyzes were carried out (using the platform) related to the influence of thermal sum and rainfall sum data (using the information collected) on the production of certain types of corn hybrids, considering the granularity by space (location) and time (period) . Finally, a dashboard was built for simple and intuitive visualization of data for analysis of past plantings by agronomists. This final view, made available in a dashboard, allows the use and consumption in a simple and intuitive way of the information that was collected and processed throughout the data pipeline. Furthermore, it is possible to make future inferences about the data, considering the existing information. Therefore, as a conclusion to the work, when creating the platform, it is clear the importance and potential of systems that can offer resilience and scalability when managing, storing, processing and analyzing large amounts of important data for the area of Agronomy. Furthermore, the importance of collecting, processing and analyzing this information stands out, with great potential for generating important insights for assertive decision.


\noindent Keywords: Agricultural production. Technology. Data analysis.
\end{otherlanguage*}


\end{resumo}