\setlength{\absparsep}{18pt} % ajusta o espaçamento dos parágrafos do resumo
\begin{resumo}
\noindent IGNAUL
IN, R.  \textbf{Protótipo de um sistema de recomendação de dados agrícolas baseado em uma plataforma de dados escalável}. 2023. Trabalho de Conclusão de Curso - Ciência da Computação, Universidade Comunitária da Região de Chapecó, Chapecó, 2023.\\


\noindent  O presente trabalho de conclusão de curso pretende a criação de uma plataforma de dados robusta para realizar análises exploratórias envolvendo a área técnica da agronomia. Para isso, foi utilizado conceitos importantes sobre arquiteturas de aplicações, além da aplicação de conceitos de computação na nuvem para processamento big data. Diversos tipos de dados foram centralizados em uma plataforma de armazenamento na nuvem robusta e escalável. A plataforma escolhida foi a AWS, o que permitiu a utilização de vários serviços que são confiáveis e altamente disponíveis na nuvem. Alguns \textit{frameworks open-source} como o Apache Spark e o Apache Airflow foram utilizados para processamento e orquestração distribuída de pipelines de dados. Foram consumidas diversas informações relacionadas a dados meteorológicos (INMET), além da coleta de informações específicas da agricultura, disponibilizada pelo (MAPA). A utilização de processos atualizados permite o processamento e análise de dados com maiores detalhes, importantes para a geração de insights. Foram realizadas diversas análises (utilizando a plataforma) relacionadas a influência dos dados de soma térmica e soma pluviométrica (utilizando as informações coletadas) na produção de determinados tipos de híbridos de milho, considerando a granularidade por espaço (localização) e por tempo (período). Por fim, foi construído um dashboard para visualização simples e intuitiva dos dados para análises de plantios passados por agrônomos.  Esta visão final, disponibilizada em um dashboard, permite a utilização e o consumo de forma simples e intuitiva das informações que foram coletadas e processadas ao longo da pipeline de dados. Além disso, é possível realizar inferências futuras sobre os dados, considerando as informações existentes. Portanto, como conclusão do trabalho, ao criar a plataforma é notado a importância e o potencial de sistemas que possam oferecer resiliência e escalabilidade na hora de gerenciar, armazenar, processar e analisar grandes quantidades de importantes dados para a área da Agronomia. Além disso, destaca-se a importância da coleta, processamento e análise dessas informações com um grande potencial para gerar insights importantes para uma tomada de decisão assertiva. 

\noindent Palavras-chave: Produção Agrícola. Tecnologia. Análise de dados.
\end{resumo}