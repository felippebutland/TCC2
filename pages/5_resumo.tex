\setlength{\absparsep}{18pt} % ajusta o espaçamento dos parágrafos do resumo
\begin{resumo}
\noindent Felippe Butland\textbf{PROTÓTIPO DE APLICAÇÃO DE ENVIO DE DOCUMENTOS MODELADO COM C4
MODEL, UTILIZANDO ARQUITETURA DE MICRO SERVIÇOS E MENSAGERIA}. 2024. Trabalho de Conclusão de Curso - Ciência da Computação, Universidade Comunitária da Região de Chapecó, Chapecó, 2024.\\


\noindent  Este trabalho descreve o desenvolvimento de um protótipo de aplicação para envio de documentos utilizando uma arquitetura de microserviços, mensageria e modelagem com o C4 Model. A escolha dessa arquitetura visa oferecer modularidade e escalabilidade, atendendo às necessidades de aplicações corporativas e à realidade de startups, que frequentemente hesitam em investir em arquitetos de software devido aos custos. Foi criada uma solução baseada em microserviços separados para login, cadastro e envio de documentos, com comunicação via mensageria, o que proporcionou um sistema eficiente e resiliente para o tráfego de dados.  Os testes de benchmark realizados demonstraram que a aplicação pode lidar com um volume satisfatório de requisições, embora a segmentação em microserviços possa introduzir latência, especialmente em ambientes multi-região. Conclui-se que uma arquitetura bem planejada e documentada, mesmo com recursos limitados, permite desenvolver aplicações escaláveis e de fácil manutenção, servindo como referência para startups e pequenas empresas que buscam eficiência e sustentabilidade a longo prazo.

\noindent Palavras-chave: Micro-Serviços, C4Model, Mensageria.
\end{resumo}